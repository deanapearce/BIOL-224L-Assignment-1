\documentclass[titlepage]{scrartcl}
\addtokomafont{disposition}{\rmfamily}
%opening
\titlehead{BIOL 224L}
\title{Assignment 1}
\date{\today}
\author{Dean Pearce}

\begin{document}

\maketitle
%\tableofcontents
%\clearpage
\part*{Interpretation}
\section{Recolonization Probability}
It would appear that the minimum recolonization probability which does not result in the eventual extinction of the plant lineage is between $ 0.0331 $ and $ 0.0332 $.
%INSERT FIGURES
\section{Grid Size}
It does not seem that the grid size has a substantial effect on the proportion of cells occupied.
\section{Connectedness}
A switch from a 8-connected neighborhood to a 4-connected neighborhood was accomplished by modification of the neighbor search distance from 2.1 to 1.05, as follows.\\\\
Original:
\begin{verbatim*}
n=find((Xp.^2+Yp.^2)<2.1);
\end{verbatim*}
Modified:
\begin{verbatim*}
n=find((Xp.^2+Yp.^2)<1.05);
\end{verbatim*}
This change does have a substantial effect upon the behavior of the model; the plant populations are much more likely to go extinct given that other parameters are unchanged. (See fig m0.05)  More broadly, it is insufficient to merely double m to restore 8-connected neighborhood behavior, as the basal assumptions of the model have been modified. (See fig m0.10)
\section{Timescale}
In this analysis, we set a value d which serves as a factor by which to dilate the timescale of the model.
\begin{verbatim}
	% Setup parameters
	<...>
	d = 1; % timescale dilator
	m = 0.05*d; % recolonization probability per each neighbour
	e = 0.16*d; % extinction probability
\end{verbatim}
AN important distinction is that such a dilator affects the amount that happens in a single timestep, i.e. an increase in the extinction / recolonization probabilities causes greater variability per timestep
\section{Reasonableness of Boundary Conditions}
\begin{quote}
	"In this simulation, we will use an absorbing boundary such that cells at the edge of	the grid do not have neighbors, and thus have a lower chance of colonization if	unoccupied."
\end{quote}
Such an assumption is, broadly, reasonable.  
\section{Significance of Edges}

\end{document}
